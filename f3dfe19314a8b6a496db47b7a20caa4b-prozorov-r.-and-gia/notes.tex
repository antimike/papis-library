\documentclass{pset}
\usepackage{macros}
\usepackage[units, chem]{include}
\usepackage{inkscape}
\usepackage{pytex-preamble}
\crefname{enumi}{}{}

\begin{document}

\author{Michael Haynes}
\title{Notes on Prozorov-Giannetta \\ Meissner-London state in superconductors of rectangular cross section in a perpendicular magnetic field} 
\date{2020-10-14}
\maketitle

\section{Weak formulation}

The solution consists of two stages:
\begin{enumerate}[label=(\Alph*)]
	\item Solve for the magnetic scalar potential $\displaystyle \Psi$, imposing the desired fluxoid quantization condition through the inhomogeneous boundary term $\displaystyle B_{QS}$ \label{first-part}
	\item Solve for the shifted vector potential $\displaystyle \vb{A}^{*}$ and the phase $\displaystyle \theta$. \label{second-part}
\end{enumerate}

Because both vector- and scalar-valued fields are involved, the finite-element function space will be a direct product of $\displaystyle Q_p(\Omega_{\tx{in}} \cup \Omega_{\tx{out}})$ (for the scalar fields) and $\displaystyle Q_p^{3}(\Omega_{\tx{in}} \cup \Omega_{\tx{out}})$ (for the vector fields).

\begin{todo}
	Investigate possible issues with convergence related to choice of finite-element spaces and the LBB condition for saddle-point problems.  I believe the above (naive) choices should be OK, but there may be better choices / I may have misunderstood the convergence criteria.
\end{todo}


\begin{definition*}[Basis functions] \label{defn:basis-functions}
	Let 
	\begin{itemize}
	  \item $\displaystyle \varphi_i$ denote the scalar basis elements
		  \item $\displaystyle \vb{\Phi}_i$ denote the vector basis elements
	\end{itemize}
\end{definition*}

\begin{definition*}[System matrices] \label{defn:system-matrices}
	The following matrices will be used to assemble the linear systems for \Cref{first-part,second-part}:
\begin{align}
	M_{ij}^{(\tx{int})} &\coloneqq \int_{ \Omega_{(\tx{int})} }^{  } \grad \varphi_i \cdot \grad \varphi_j \dd{\Omega} \label{Mij-int} \\
	M_{ij}^{(\tx{ext})} & \coloneqq \int_{ \Omega_{\tx{(ext)}} }^{  } \grad \varphi_i \cdot \grad \varphi_j \dd{\Omega} \label{Mij-ext} \\
	C_{ij}^{\tx{(int/ext)}} & \coloneqq \int_{ \Omega_{\tx{(int/ext)}} }^{  } (\grad \times \vb{\Phi}_i) \cdot (\grad \times \vb{\Phi}_j) \dd{\Omega} \label{Cij-int/ext} \\
		D_{ij}^{\tx{(int/ext)}} & \coloneqq \int_{ \Omega_{\tx{(int/ext)}} }^{  } (\grad \cdot \vb{\Phi}_i)(\grad \cdot \vb{\Phi}_j) \dd{\Omega} \label{Sij-int/ext} \\
					S_{ij}^{\tx{(int/ext)}} & \coloneqq \int_{ \Omega_{\tx{(int/ext)}} }^{  } \vb{\Phi}_i \cdot \vb{\Phi}_j \dd{\Omega} \label{Dij-int/ext} \\
						T_{ij}^{\tx{(int/ext)}} & \coloneqq \int_{ \Omega_{\tx{(int/ext)}} }^{  } \vb{\Phi}_i \cdot \grad \varphi_j \dd{\Omega} \label{Tij-int/ext} \\
							E_{ij}^{\tx{(ext)}} & \coloneqq \int_{ \Omega_{\tx{(ext)}} }^{  } (\grad \times \vb{\Phi}_i) \cdot \grad \varphi_j \dd{\Omega} \label{Eij-ext} \\
							H_i^{\tx{(ext)}} & \coloneqq \int_{ \Gamma_H }^{  } \vb{\Phi}_i \cdot \vb{H}_{\tx{(ext)}} \dd{\Omega} \label{H-ext} \\
							B_i^{\tx{(QS)}} & \coloneqq \int_{ S_a }^{  } \varphi_i B_{QS} \dd{S} \label{Bi} 
\end{align}
	
\end{definition*}

The weak formulation of the problem is then
\begin{gather}
	M_{ij}^{\tx{(ext)}} \vb{\Psi}_j = B^{(\tx{QS})}_i \tag*{\autoref{first-part}} \\
	\begin{dcases}
		& \left[ C_{ij}^{(\tx{int})} + pD_{ij}^{(\tx{int})} + \dfrac{1}{\lambda^2} S_{ij}^{(\tx{int})} \right] A^*_{j} = \dfrac{\Phi_0}{2\pi \lambda^2} T_{ij}^{(\tx{int})} \theta_i \\
		& \dfrac{1}{\mu_0 \lambda^2} \til{T}_{ij}^{(\tx{int})} A_j^* = \dfrac{\Phi_0}{2\pi \mu_0 \lambda^2} M_{ij}^{(\tx{int})} \theta_i \\
		& \left[ C_{ij}^{(\tx{ext})} + pD_{ij}^{(\tx{ext})} \right] A^*_i = -E_{ij}^{(\tx{ext})} \Psi_j + \vb{H}_i^{(\tx{ext})}
	\end{dcases} \tag*{\autoref{second-part}}	
\end{gather}




\end{document}
